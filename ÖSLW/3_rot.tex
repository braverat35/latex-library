\section{ROT -- Fortgeschrittene parallele Technik}

% ----------------------
\subsection{Paralleles Skisteuern lang}

\textbf{Form:} Parallele Schwünge in langen Radien.\\
\textbf{Ziel:} Gleichzeitiges Umkanten, Drehen und Steuern der Skier.\\
\textbf{Inhalt:}
\begin{itemize}[noitemsep]
  \item parallele Skiführung während des gesamten Schwungs
  \item dynamische Hoch-Tiefbewegung
  \item Rhythmus über vorbereitenden Stockeinsatz
\end{itemize}
\textbf{Übungen:}
\begin{itemize}
  \item \textbf{Lang-kurz-lang:} Mehrere lange parallele Schwünge, dann einige kürzere, wieder lange; Fokus auf gleiche Bewegungsabfolge.
  \item \textbf{Bahnmarkierung:} Mit Hütchen eine Ideallinie markieren und lange parallele Schwünge möglichst genau auf dieser Linie fahren.
  \item \textbf{Tempovariation:} Gleiche Radien fahren, aber einmal in langsamem, einmal in etwas höherem Tempo, um Bewegungskonstanz zu trainieren.
\end{itemize}

% ----------------------
\subsection{Stockeinsatz}

\textbf{Ziel:} Hilfe für Gleichgewicht, Entlastung, Drehen und Rhythmus.\\
\textbf{Inhalt:}
\begin{itemize}[noitemsep]
  \item präziser, kleiner Stockeinsatz vor dem Kantenwechsel
  \item keine Rotation aus der Schulter
  \item Rhythmusgebung für den Schwung
\end{itemize}
\begin{itemize}
  \item \textbf{Trockenübung im Stand:} Stockeinsatz ohne Fahrtbewegung üben: Einsatzort, Handhaltung und Bewegungsrichtung klären.
  \item \textbf{Schuss mit Stockimpuls:} In sanfter Schussfahrt den Stockeinsatz im Rhythmus simulieren, ohne Kurven zu fahren, um das Timing zu spüren.
  \item \textbf{Langsame Schwünge mit Fokus Stock:} Weite parallele Schwünge fahren, bei denen jeder Stockeinsatz bewusst vorbereitet und exakt vor dem Kantenwechsel gesetzt wird.
\end{itemize}

% ----------------------
\subsection{Paralleles Skisteuern -- lange Radien mit Stockeinsatz}

\textbf{Form:} Parallele Schwünge, mittlere bis lange Radien.\\
\textbf{Ziel:} Stabiler Rhythmus und klare Schwungeinleitung.\\
\textbf{Inhalt:}
\begin{itemize}[noitemsep]
  \item höheres Tempo möglich
  \item flüssige Schwünge mit Unterstützung durch den Stockeinsatz
\end{itemize}
\begin{itemize}
  \item \textbf{Metronom-Schwung:} Schwünge im Takt zählen (eins-zwei, eins-zwei) und Stockeinsatz jeweils auf die Zählzeit einbauen.
  \item \textbf{Wellenbahn mit Stockeinsatz:} Über eine leicht gewellte Piste fahren und den Stockeinsatz mit den natürlichen Geländebewegungen koordinieren.
  \item \textbf{Folgefahren:} In einer Linie hinter dem Lehrer fahren und dessen Stockeinsatz sowie Radius und Rhythmus möglichst genau imitieren.
\end{itemize}

% ----------------------
\subsection{Paralleles Skisteuern -- kurze Radien}

\textbf{Form:} Rhythmische kurze parallele Schwünge.\\
\textbf{Ziel:} Schnelle Richtungswechsel und präzise Steuerung.\\
\textbf{Inhalt:}
\begin{itemize}[noitemsep]
  \item kompakte Position
  \item schneller Druckwechsel
  \item Stockeinsatz für Timing
\end{itemize}
\textbf{Übungen:}
\begin{itemize}
  \item \textbf{Pendelschwünge:} Auf kleinem Hangabschnitt mehrere sehr kurz aneinandergereihte Schwünge mit möglichst gleichbleibendem Rhythmus fahren.
  \item \textbf{Spurbrett:} Auf einer schmal vorgegebenen Spur (z.\,B. durch Spuren im Schnee) kurze parallele Schwünge reißen, ohne die Spur zu verlassen.
  \item \textbf{Rhythmuswechsel:} Abwechselnd 4 mittellange und 4 kurze Schwünge fahren, um das Umstellen der Bewegungsfrequenz zu trainieren.
\end{itemize}

% ----------------------
\subsection{Paralleles Skisteuern dynamisch -- lange Radien}

\textbf{Form:} Sportlich dynamische Parallelschwünge.\\
\textbf{Ziel:} Situationsgerechtes, kraftvolles Skifahren in mittleren und langen Radien.\\
\textbf{Inhalt:}
\begin{itemize}
  \item großer Kantwinkel
  \item dynamische Hoch-Tiefbewegung
  \item Nutzung der Reboundkräfte
\end{itemize}
\textbf{Übungen:}
\begin{itemize}
  \item \textbf{Dynamikaufbau:} Mehrmals denselben Hangabschnitt fahren und bei jeder Wiederholung etwas mehr Kantwinkel, Bewegungsumfang und Tempo einsetzen.
  \item \textbf{Spurziehen dynamisch:} In einer breiten Piste eine deutliche Spur (Bahn) legen, bei der die Radien gleichbleibend aber die Dynamik erhöht wird.
  \item \textbf{Geländeübergänge:} Übergänge von flacherem in etwas steileres Gelände bewusst nutzen, um dynamische Schwünge mit zunehmendem Druck aufzubauen.
\end{itemize}

% ----------------------
\subsection{Paralleles Skisteuern dynamisch -- kurze Radien}

\textbf{Form:} Dynamische Kurzschwünge.\\
\textbf{Ziel:} Kraftvolle, enge Schwünge in anspruchsvollerem Gelände.\\
\textbf{Inhalt:}
\begin{itemize}
  \item hohe Bewegungsfrequenz
  \item präziser Stockeinsatz
  \item starke Kantwinkel im Kurvenmittelteil
\end{itemize}
\textbf{Übungen:}
\begin{itemize}
  \item \textbf{Hüpf-Kurzschwung:} Am Schwungwechsel kleine Sprünge einbauen, um die Entlastungsphase zu betonen, danach wieder ohne Sprung dieselbe Dynamik halten.
  \item \textbf{Stangen- oder Hütchenlinie:} In enger Linie aufgestellte Hütchen im dynamischen Kurzschwung umfahren, Fokus auf sauberen Rhythmus.
  \item \textbf{Steilerer Hangabschnitt:} Auf einem etwas steileren Stück gezielt kurze, dynamische Schwünge mit klarer Entlastung und Markierung der Schwungpunkte fahren.
\end{itemize}