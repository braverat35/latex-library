\section{SCHWARZ -- Expertenniveau / Spezialformen}

% ----------------------
\subsection{Carven -- lange Radien}

\textbf{Form:} Geschnittene Kurven ohne Drift.\\
\textbf{Ziel:} Sportlich sicheres Carven in langen Radien. Vorbereitung Riesenslalom.\\
\textbf{Inhalt:}
\begin{itemize}[noitemsep]
  \item früher Kantaufbau
  \item hoher Kantwinkel
  \item klar geschnittene Linie
\end{itemize}
\textbf{Übungen:}
\begin{itemize}[noitemsep]
  \item \textbf{Spurziehen auf blauer Piste:} Auf flacher bis mittelsteiler Piste geschnittene Schwünge fahren und auf eine durchgehende, klare Spur achten.
  \item \textbf{Carving-Korridor:} Einen Korridor mit Stangen markieren und die Carvingschwünge so anlegen, dass die Spur komplett im Korridor bleibt.
  \item \textbf{Tempovariation im Carve:} Gleiche Radien carven, aber einmal in moderatem und einmal in höherem Tempo, ohne ins Driften zu kommen.
\end{itemize}

% ----------------------
\subsection{Carven -- kurze Radien}

\textbf{Form:} Geschnittene Richtungsänderungen in kurzen Radien.\\
\textbf{Ziel:} Präzises Slalomcarven.\\
\textbf{Inhalt:}
\begin{itemize}[noitemsep]
  \item sehr schnelle Umkantphase
  \item hohe Stabilität im Rumpf
  \item enger Kurvenradius
\end{itemize}
\textbf{Übungen:}
\begin{itemize}[noitemsep]
  \item \textbf{Kurze Carves um Hütchen:} In enger Abfolge Hütchen ausstecken und mit kurzen, möglichst geschnittenen Schwüngen umfahren.
  \item \textbf{Innen- und Außentor-Linie:} Zwei enge Linien stecken (innen/außen) und abwechselnd engere und etwas weitere Carves fahren, ohne zu driften.
  \item \textbf{Zeitfenster:} Kurze Carvingradien mit bewusst frühem Kantaufbau, sodass der Großteil der Richtungsänderung im oberen Teil des Schwunges passiert.
\end{itemize}

% ----------------------
\subsection{Steilhänge}

\textbf{Ziel:} Sicheres Befahren von steilem Gelände.\\
\textbf{Inhalt:}
\begin{itemize}[noitemsep]
  \item kompakter Schwerpunkt
  \item kurze Radien
  \item Tempokontrolle durch Rhythmus
\end{itemize}
\textbf{Übungen:}
\begin{itemize}[noitemsep]
  \item \textbf{Treppenschwünge:} Hangquerung, kurze Drehung zur Falllinie, sofort wieder Querfahrt; Schritt für Schritt nach unten arbeiten.
  \item \textbf{Einstieg vom flacheren in steileren Teil:} Zuerst auf flacherem Teil kurze Radien fahren, dann kontrolliert in steileren Abschnitt übergehen, Rhythmus beibehalten.
  \item \textbf{Punkt-Schwünge:} Ein optisches Ziel im steileren Hang fixieren und dort den Schwungpunkt setzen, danach wieder breiter ausfahren.
\end{itemize}

% ----------------------
\subsection{Buckel}

\textbf{Form:} Beugendes und streckendes Drehen.\\
\textbf{Ziel:} Kontrolliertes Befahren der Buckel bei ständigem Schneekontakt.\\
\textbf{Inhalt:}
\begin{itemize}[noitemsep]
  \item schnelle Beuge-Streckbewegung
  \item Ski im Bodenkontakt halten
  \item kurze Radien über den Buckeln
\end{itemize}
\textbf{Übungen:}
\begin{itemize}[noitemsep]
  \item \textbf{Einzelbuckel-Annäherung:} Zunächst einzelne, gut sichtbare Buckel anfahren und über diese gezielt beugen und strecken.
  \item \textbf{Zwei-Buckel-Rhythmus:} Jeweils über zwei aufeinanderfolgende Buckel eine Schwungkombination legen (Buckel eins beugen, Buckel zwei strecken).
  \item \textbf{Linienwahl:} Oberhalb der Buckelrücken fahren und bewusst kurze Schwünge auf der Buckeloberkante anlegen.
\end{itemize}

% ----------------------
\subsection{Tiefschnee}

\textbf{Ziel:} Rhythmisches, genussvolles Skifahren im weichen Schnee.\\
\textbf{Inhalt:}
\begin{itemize}[noitemsep]
  \item zentraler Stand
  \item federnde Bewegung
  \item gebundener Rhythmus für Auftrieb
\end{itemize}
\textbf{Übungen:}
\begin{itemize}[noitemsep]
  \item \textbf{Rhythmische Federbewegung:} Auf moderat tiefem Schnee federnde Beuge-Streckbewegungen im gleichen Rhythmus durchführen, bevor Schwünge eingebaut werden.
  \item \textbf{Spur des Lehrers nachfahren:} Der Lehrer legt eine Spur im Tiefschnee an, die Gruppe folgt exakt dieser Linie mit gleichen Schwungpunkten.
  \item \textbf{Kleine Geländeübergänge:} Leicht variierendes Gelände (Mulden, Wellen) nutzen, um den \enquote{Auftrieb-Abtrieb}-Rhythmus zu spüren.
\end{itemize}

% ----------------------
\subsection{Demo \& Formation}

\textbf{Ziel:} Synchrone Richtungsänderungen und Bewegungen in der Gruppe.\\
\textbf{Inhalt:}
\begin{itemize}[noitemsep]
  \item gleiche Radien
  \item gleiche Geschwindigkeit
  \item abgestimmte Bewegungsabläufe
\end{itemize}
\textbf{Übungen:}
\begin{itemize}[noitemsep]
  \item \textbf{Zweier-Synchronfahren:} In Zweierreihen nebeneinander fahren und versuchen, Schwünge und Bewegungen exakt zeitgleich auszuführen.
  \item \textbf{Formationswechsel:} Aus einer Reihe in eine V-Formation und wieder zurück fahren, ohne den Rhythmus der Schwünge zu verlieren.
  \item \textbf{Spurüberlagerung:} Ein Fahrer gibt die Spur vor, die folgenden fahren exakt in dieselbe Linie, sodass nur eine \enquote{gemeinsame} Spur sichtbar bleibt.
\end{itemize}

% ----------------------
\subsection{Freeriden}

\textbf{Ziel:} Sicheres Fahren in freiem Gelände aller Steilheiten.\\
\textbf{Inhalt:}
\begin{itemize}[noitemsep]
  \item vorausschauendes Fahren
  \item Anpassung von Rhythmus und Radius
  \item Linienwahl nach Gelände und Schnee
\end{itemize}
\textbf{Übungen:}
\begin{itemize}[noitemsep]
  \item \textbf{Gelände lesen:} Vor einer Abfahrt kurz anhalten, mögliche Linien im Gelände besprechen und anschließend bewusst eine gewählte Linie fahren.
  \item \textbf{Linienvariation:} Mehrmals denselben Hang fahren, jedes Mal eine andere Linie wählen (breitere Schwünge, enger, direkter), um Anpassungsfähigkeit zu schulen.
  \item \textbf{Stopp-Punkte:} Im Gelände markierte Punkte (Felsen, Bäume, Geländeformen) als Stopps oder Sammelpunkte wählen und darauf hin vorausschauend fahren.
\end{itemize}