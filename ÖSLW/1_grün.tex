\section{GRÜN -- Anfängerbereich}

% ----------------------
\subsection{Gewöhnen an die Skiausrüstung}

\textbf{Form:} Gehen, Gleiten und Aufsteigen in sanft gegliedertem Gelände.\\
\textbf{Ziel:} Gewöhnung an die Ausrüstung und an den eingeschränkten Bewegungsumfang. Sicherheit und Orientierung im flachen Gelände.\\
\textbf{Inhalt:}
\begin{itemize}[noitemsep]
  \item Schritte mit Ski, Drehen, Gehen, Aufsteigen
  \item Gleichgewicht im Stand und während erster Gleitphasen
  \item kontrolliertes Bewegen im flachen Gelände
\end{itemize}
\textbf{Übungen:}
\begin{itemize}[noitemsep]
  \item \textbf{Partner-Schubsen:} Zwei stehen sich gegenüber, einer schubst leicht an Schultern oder Hüfte, der andere versucht stabil in Grundposition zu bleiben.
  \item \textbf{Storch:} Im Stand jeweils einen Ski leicht anheben (vorne, seitlich, hinten), Gleichgewicht halten und Position wechseln.
  \item \textbf{Hände auf Knie / Helm:} Beim Gehen mit Ski wechseln die Teilnehmenden die Handposition (an die Knie, an die Hüfte, an den Helm), ohne aus dem Gleichgewicht zu kommen.
\end{itemize}

% ----------------------
\subsection{Gleit- und Schussübungen}
\textbf{Form:} Geradeausgleiten / Schussfahrt.\\
\textbf{Ziel:} Gleichgewicht halten im Gleiten.\\
\textbf{Inhalt:}
\begin{itemize}[noitemsep]
  \item Schussfahrt in Grundposition
  \item stabile Haltung und Spurkontrolle
  \item leichte Variationen (Hände in verschiedenen Positionen, Hocke)
\end{itemize} 
\textbf{Übungen:}
\begin{itemize}[noitemsep]
  \item \textbf{Schuss mit Hocke und Aufrichten:} Aus der Hocke losfahren, langsam in die Grundposition aufrichten und wieder in die Hocke gehen.
  \item \textbf{Arme-Variationen:} Geradeausgleiten mit Armen nach vorne, seitlich, über dem Kopf; Fokus auf ruhigen Oberkörper.
  \item \textbf{Hasenhüpfer:} In der Schussfahrt kleine, gleichmäßige Sprünge machen, ohne die Spur zu verlieren.
\end{itemize}
% ----------------------
\subsection{Pflug}

\textbf{Form:} Schneepflug (A-Form der Ski).\\
\textbf{Ziel:} Gleiten im Pflug, Bremsen und Anhalten.\\
\textbf{Inhalt:}
\begin{itemize}[noitemsep]
  \item Pflugstellung herstellen
  \item Geschwindigkeit regulieren (Pflug öffnen/schließen)
  \item sichere Brems- und Anhaltefähigkeit
\end{itemize}
\textbf{Übungen:}
\begin{itemize}[noitemsep]
  \item \textbf{Punktbremsen:} Auf ein sichtbares Ziel zufahren (Hütchen, Stock) und genau dort im Pflug zum Stehen kommen.
  \item \textbf{Pizza-Pommes:} Abwechselnd kurze Stücke im Pflug (Pizza) und in paralleler Spur (Pommes) fahren, auf Zuruf des Lehrers wechseln.
  \item \textbf{Pflug-Korridor:} Durch einen mit Stangen oder Hütchen markierten Korridor im Pflug fahren, ohne die Markierungen zu berühren.
  \item \textbf{Bremslokomotive:} Paarweise mit Seilen verbunden hintereinander fahren. Die hintere Person versucht durch einen höheren Aufkantwinkel die vordere Person abzubremsen.
\end{itemize}