\section{BLAU -- Grundtechnik}

% ----------------------
\subsection{Pflugdrehen}

\textbf{Form:} Richtungsänderungen im Pflug.\\
\textbf{Ziel:} Kurvenfahren im Schneepflug.\\
\textbf{Inhalt:}
\begin{itemize}[noitemsep]
  \item Außenskibelastung
  \item Kurveneinleitung über Blick und Haltung
  \item Tempokontrolle über Pflugwinkel
\end{itemize}
\textbf{Übungen:}
\begin{itemize}[noitemsep]
  \item \textbf{Pflug-Slalom:} In weiten Bögen um Hütchen oder Fähnchen im Pflug fahren, Kurven sauber zu Ende fahren.
  \item \textbf{Zielkurven:} Auf ein Ziel seitlich der Falllinie schauen und gezielt mit Pflugkurve dorthin fahren.
  \item \textbf{Kurven auf Zuruf:} Lehrer ruft links oder rechts, die Gruppe fährt sofort eine Pflugkurve in die entsprechende Richtung.
\end{itemize}

% ----------------------
\subsection{Alpines Fahrverhalten -- Kanten}

\textbf{Form:} Fahren auf den bergseitigen Kanten.\\
\textbf{Ziel:} Kantenwirkung spüren und Seitenführung entwickeln.\\
\textbf{Inhalt:}
\begin{itemize}[noitemsep]
  \item Schrägfahrt bergseitig
  \item Knie und Hüfte talwärts
  \item stabile Körperposition
\end{itemize}
\textbf{Übungen:}
\begin{itemize}[noitemsep]
  \item \textbf{Schrägfahrt halten:} In leichter Schrägfahrt bergseitige Kanten bewusst einsetzen und die Spur ohne Höhenverlust halten.
  \item \textbf{Kantenwechsel im Geradeausfahren:} In der Schussfahrt das Gewicht von linker auf rechte Kante verlagern, ohne zu drehen.
  \item \textbf{Schrägfahrt über Wellen:} Schrägfahrt über leicht unebenes Gelände, dabei Kantendruck und Körperposition stabil halten.
\end{itemize}

% ----------------------
\subsection{Alpines Fahrverhalten -- Rutschen}

\textbf{Form:} Parallel geführte Ski im Rutschen steuern.\\
\textbf{Ziel:} Kantengriff lösen, kontrolliert rutschen.\\
\textbf{Inhalt:}
\begin{itemize}[noitemsep]
  \item Ski flach stellen
  \item kontrolliertes Abrutschen in der Falllinie
  \item Kante schließen zum Beenden des Rutschens
\end{itemize}
\textbf{Übungen:}
\begin{itemize}[noitemsep]
  \item \textbf{Seitliches Abrutschen:} In Hanglage die Ski flach stellen und seitlich abrutschen lassen, anschließend Kanten wieder schließen.
  \item \textbf{Rutschen und Stoppen:} In der Falllinie anfahren, Ski leicht querstellen, rutschen lassen und durch stärkeres Kanten bewusst stoppen.
  \item \textbf{Rutsch-Girlande:} Schrägfahrt mit kurzen rutschenden Richtungsänderungen bergauf/bergab, ohne ganze Schwünge zu fahren.
\end{itemize}

% ----------------------
\subsection{Pflugsteuern}

\textbf{Form:} Richtungsänderungen mit Steuern auf den bergseitigen Kanten (Schwungende wird parallel).\\
\textbf{Ziel:} Übergang zur parallelen Skiführung.\\
\textbf{Inhalt:}
\begin{itemize}[noitemsep]
  \item Innenski entlasten
  \item Außenskidruck verstärken
  \item Pfluganteil nimmt ab, parallele Phase nimmt zu
\end{itemize}
\textbf{Übungen:}
\begin{itemize}[noitemsep]
  \item \textbf{Pflug-Parallel-Ende:} Pflugsteuerkurven fahren und bewusst am Schwungende die Ski schließen, sodass sie parallel werden.
  \item \textbf{Innenfuß heben:} Während des Pflugsteuerns den Innenski kurz anheben, um Außenskibelastung zu betonen.
  \item \textbf{Pflugsteuern im Slalom:} Leicht versetzte Hütchen vorgeben, Kurven im Pflug beginnen und mit paralleler Skiführung zwischen den Hütchen beenden.
\end{itemize}